\documentclass[10pt,a4paper]{article}
\usepackage[utf8]{inputenc}
\usepackage{amsmath}
\usepackage{amsfonts}
\usepackage{amssymb}
\usepackage{graphicx}
\author{Frode T. Børseth}
\date{\today}
\title{Numerical exercise\\FY2045 Quantum Mechanics}
\begin{document}
\maketitle

\section{Introduction}
The one-dimensional Schr\"odinger equation in position space of a particle with mass $m$ subject to a potential $V$ reads
\begin{equation}
	\mathrm{i} \hbar \frac{\partial}{\partial t}
	\Psi ( x, t)
	=
	- \frac{\hbar^2}{2m} \frac{\partial^2}{\partial x^2}
	\Psi (x, t)
	+
	V(x)
	\Psi (x, t) \label{eq:Sch}
\end{equation}
and is the object of interest in this exercise.

The wave function is generally complex, crucially so when considering its time evolution, and we may consider the real and imaginary components distinct functions,
\begin{equation}
	\Psi = \Psi_R + \mathrm{i} \Psi_I,
\end{equation}
all of which in general are functions of position $x$ and time $t$. The function $\Psi_I$ in particular is real valued. Inserting this into equation \ref{eq:Sch} gives
\begin{align*}
	\mathrm{i} \hbar \frac{\partial}{\partial t}
	\Big(
	\Psi_R + \mathrm{i}\Psi_I
	\Big)
	&=
	- \frac{\hbar^2}{2m} \frac{\partial^2}{\partial x^2}
	\Big(
	\Psi_R + \mathrm{i}\Psi_I
	\Big)
	+
	V
	\Big(
	\Psi_R + \mathrm{i}\Psi_I
	\Big)\\
	\Rightarrow \;
	- \hbar \frac{\partial}{\partial t}
	\Psi_I
	+
	\mathrm{i} \hbar \frac{\partial}{\partial t}
	\Psi_R
	&=
	- \frac{\hbar^2}{2m} \frac{\partial^2}{\partial x^2}
	\Psi_R
	+
	V\Psi_R 
	+
	\mathrm{i}
	\Big(
	-
	\frac{\hbar^2}{2m} \frac{\partial^2}{\partial x^2}
	\Psi_I
	+ 
	\mathrm{i}
	V\Psi_I
	\Big)
\end{align*}

This can then be separated into two equations, one for the real component and one for the imaginary:
\begin{align}
	- \hbar \frac{\partial}{\partial t} \Psi_I
	&=
	- \frac{\hbar^2}{2m} \frac{\partial^2}{\partial x^2}
	\Psi_R
	+
	V
	\Psi_R\\
	\hbar \frac{\partial}{\partial t} \Psi_R
	&=
	- \frac{\hbar^2}{2m} \frac{\partial^2}{\partial x^2}
	\Psi_I
	+
	V
	\Psi_I
\end{align}
This is a set of partial differential equations, containing the \emph{real} functions $\Psi_R$ and $\Psi_I$.

Our strategy will be do discretize the problem. First of all, restrict the particle to the region $0 \leq x \leq L$. Then, evaluate the wave function at $N_x+1$ points on this interval, and use these points and values of $\Psi$ to approximate the continuous solution.

The discretization leads to a one-dimensional mesh of $x$ values, a distance $\Delta x$ apart, which can be expressed as $x_n = n\Delta x$. The same applies for time, which changes step by step in increments of $\Delta t$. The different discrete times at which we evaluate $\Psi$ are $t_m = m\Delta t$.

Let us now for simplicity write the different functions at each discrete point and time in the following way:
\begin{align}
	\Psi_R^{n,m} &= \Psi_R(x_n,t_m) = \Psi_R(n\Delta x,m\Delta t)\\
	\Psi_I^{n,m} &= \Psi_I(x_n,t_m) = \Psi_I(n\Delta x,m\Delta t)\\
	V^n &= V(x_n) = V(n\Delta x) \quad \quad \text{(assuming non-varying potential)}
\end{align}
With this, we can write down the discretized partial derivatives as
\begin{align}
	\frac{\partial \Psi_F}{\partial t} (x_n,t_m)
	&\approx
	\frac{1}{\Delta t} \Big( \Psi_F^{n,m+1} - \Psi_F^{n,m}\Big)\\
	\frac{\partial^2 \Psi_F}{\partial x^2} (x_n,t_m)
	&\approx
	\frac{1}{\Delta x^2} \Big( \Psi_F^{n+1,m} - 2\Psi_F^{n,m} + \Psi_F^{ n-1,m} \Big)\\
\end{align}
leading to a discrete version of the above system of equations,
\begin{align}
	- 
	\frac{\hbar}{\Delta t} \Big( \Psi_I^{n,m+1} - \Psi_I^{n,m}\Big)
	&=
	- \frac{\hbar^2}{2m}
	\frac{1}{\Delta x^2}
	\Big(
		\Psi_R^{n+1,m} - 2\Psi_R^{n,m} + \Psi_R^{ n-1,m}
	\Big)
	+
	V^n
	\Psi_R^{n,m}\\
	\frac{\hbar}{\Delta t} \Big( \Psi_R^{n,m+1} - \Psi_R^{n,m}\Big)
	&=
	- \frac{\hbar^2}{2m}
	\frac{1}{\Delta x^2}
	\Big(
		\Psi_I^{n+1,m} - 2\Psi_I^{n,m} + \Psi_I^{ n-1,m}
	\Big)
	+
	V^n
	\Psi_I^{n,m}
\end{align}
which in turn, finally, gives us a way to see how this system changes from  $t_m$ to $t_{m+1}$:
\begin{align}
	\Psi_I^{n,m+1}
	&=
	\Psi_I^{n,m}
	+
	\frac{\hbar}{2m}
	\frac{\Delta t}{\Delta x^2}
	\Big(
		\Psi_R^{n+1,m} - 2\Psi_R^{n,m} + \Psi_R^{ n-1,m}
	\Big)
	+
	\frac{\Delta t}{\hbar}
	V^n
	\Psi_R^{n,m}\\
	\Psi_R^{n,m+1}
	&=
	\Psi_R^{n,m}
	- \frac{\hbar}{2m}
	\frac{\Delta t}{\Delta x^2}
	\Big(
		\Psi_I^{n+1,m} - 2\Psi_I^{n,m} + \Psi_I^{ n-1,m}
	\Big)
	+
	\frac{\Delta t}{\hbar}
	V^n
	\Psi_I^{n,m}
\end{align}

Now I want to rephraze this iteration scheme in the language of matrix vectors and products. Let us define a sequence of vectors $\psi_R^{m}$,
\begin{equation}
	\psi_R^{m} = ( \Psi_R^{0,m}, \ldots , \Psi_R^{N_x + 1,m} ) ^T
\end{equation}
that is, the vectors $\psi_R^0, \psi_R^1, \ldots , \psi_R^m, \ldots$ containing in index $n$ the value of $\Psi_R^{n,m}$. Then, for the tridiagonal matrix $K$,
\begin{equation}
	K
	=
	\left(
	\begin{matrix}
		-2 &  1 &   &\\
		 1 & -2 & \ddots &\\
		   &  \ddots & \ddots &1\\
		   &    & 1 & -2
	\end{matrix}
	\right)
\end{equation}
we may write the parenthesis above as the $n$th element of the matrix product $K\psi_R^m$.
\begin{equation}
	\Big(K \psi_R^m\Big)_n
	=
	\Psi_R^{n+1,m} - 2\Psi_R^{n,m} + \Psi_R^{ n-1,m}
\end{equation}
Similarly, defining the matrix $P$,
\begin{equation}
	P
	=
	\left(
	\begin{matrix}
		V^0 &  &   &\\
		 & V^1 &  &\\
		   &  & \ddots &\\
		   &    & & V^{N_x + 1}
	\end{matrix}
	\right)
\end{equation}
we may write $V^n\Psi_R^{n,m}$ as a matrix product as well, as the $n$th element of the vector given by $P \psi_R^m$
\begin{equation}
	\Big(P \psi_R^m\Big)_n
	=
	V^n\Psi_R^{n,m}
\end{equation}
Inserting this, I will rewrite the above relations between one time and the next as matrix relations.
\begin{align}
	\psi_I^{m+1}
	&=
	I \psi_I^m
	+
	\frac{\hbar}{2m}\frac{\Delta t}{\Delta x ^2}
	K \psi_R^m
	+
	\frac{\Delta t}{\hbar}
	P \psi_R^m\\
	\psi_R^{m+1}
	&=
	I \psi_R^m
	-
	\frac{\hbar}{2m}\frac{\Delta t}{\Delta x ^2}
	K \psi_I^m
	+
	\frac{\Delta t}{\hbar}
	P \psi_I^m
\end{align}
or, defining new scaled matrices
\begin{align}
	D &= \frac{\hbar}{2m}\frac{\Delta t}{\Delta x ^2} K\\
	V &= \frac{\Delta t}{\hbar} P
\end{align}
we can write this in a shorter way,
\begin{align}
	\psi_I^{m+1}
	&=
	I \psi_I^m
	+
	D \psi_R^m
	+
	V \psi_R^m\\
	\psi_R^{m+1}
	&=
	I \psi_R^m
	-
	D \psi_I^m
	+
	V \psi_I^m
\end{align}

This is a set of two linear equations with two unknown vectors. If we think of $\psi_R^{m+1}$ and $\psi_I^{m+1}$ as actually being one vector, with the two vectors standing on top of eachother, we can rewrite the whole thing as one giant matrix equation:
\begin{equation}
	\left(
	\begin{matrix}
	\psi_R^{m+1} \\
	\psi_I^{m+1}
	\end{matrix}
	\right)
	=
	\left(
	\begin{matrix}
	I   & V-D\\
	V+D & I
	\end{matrix}
	\right)
	\cdot
	\left(
	\begin{matrix}
	\psi_R^{m} \\
	\psi_I^{m}
	\end{matrix}
	\right)
\end{equation}

We might have done this slightly differently, however. If we had evaluated the right hand side of the Schr\"odinger equation at time $t_{m+1}$ instead of $t_m$, we would have gotten
\begin{align}
	\psi_I^{m+1}
	-
	D \psi_R^{m+1}
	-
	V \psi_R^{m+1}
	&=
	I \psi_I^m\\
	\psi_R^{m+1}
	+
	D \psi_I^{m+1}
	-
	V \psi_I^{m+1}
	&=
	I \psi_R^m
\end{align}
Leading to a slightly different final matrix equation.
\begin{equation}
	\left(
	\begin{matrix}
		I   & D-V\\
		-D-V & I
	\end{matrix}
	\right)
	\cdot
	\left(
	\begin{matrix}
		\psi_R^{m+1} \\
		\psi_I^{m+1}
	\end{matrix}
	\right)
	=
	\left(
	\begin{matrix}
		\psi_R^{m} \\
		\psi_I^{m}
	\end{matrix}
	\right)
\end{equation}
To solve this for the next, $(m+1)$th, wave vector, we must here take the inverse of the matrix on the left. Fortunately, as long as the potential does not vary, we only need to do this once.
\begin{equation}
	\left(
	\begin{matrix}
		\psi_R^{m+1} \\
		\psi_I^{m+1}
	\end{matrix}
	\right)
	=
	\left(
	\begin{matrix}
		I   & D-V\\
		-D-V & I
	\end{matrix}
	\right)^{-1}
	\cdot
	\left(
	\begin{matrix}
		\psi_R^{m} \\
		\psi_I^{m}
	\end{matrix}
	\right)
\end{equation}
I will test both approaches to see if there is any difference in stability.

\begin{align*}
f(x,y) \quad = \quad
\Bigg(
\left( 
\frac{11.6 - x}{5} 
\right)^{10} 
& \Big( 
 \exp \left( - 10 \left(1.879(x-10.2)^2 -0.644(x-10.2)(y-13.0) + 1.395(y-13.0)^2  \right)^{10} \right)
\\ &+ 
 \exp \left( - 10 \left(0.826(x-9.1)^2 + 0.341(x-9.1)(y-11.3) + 0.308(y-11.3)^2  \right)^{16} \right)
\\ &+ 
 \exp \left( - 10 \left(1.374(x-9.2)^2 + 0.186(x-9.2)(y-10.0) + 0.453(y-10.0)^2  \right)^{10} \right)
\\ &+ 
 \exp \left( - 10 \left(1.140(x-9.4)^2 -0.229(x-9.4)(y-8.4) + 1.094(y-8.4)^2  \right)^{10} \right)
\\ &+ 
 \exp \left( - 10 \left(1.562(x-9.35)^2 -0.393(x-9.35)(y-7.0) + 0.416(y-7.0)^2  \right)^{10} \right)
\\ &+ 
 \exp \left( - 10 \left(6.925(x-9.1)^2 -6.037(x-9.1)(y-11.95) + 1.777(y-11.95)^2  \right)^{10} \right)
 \Big) 
\\ + 
\left( 
\frac{9.6 - x + 0.3y}{5} 
\right)^{10} 
& \Big( 
 \exp \left( - 10 \left(2.777(x-8.85)^2 -1.418(x-8.85)(y-5.1) + 1.0(y-5.1)^2  \right)^{10} \right)
\\ &+ 
 \exp \left( - 10 \left(6.25(x-8.6)^2 -4.862(x-8.6)(y-3.9) + 1.562(y-3.9)^2  \right)^{10} \right)
\\ &+ 
 \exp \left( - 10 \left(7.304(x-8.0)^2 -5.752(x-8.0)(y-2.6) + 6.25(y-2.6)^2  \right)^{10} \right)
\\ &+ 
 \exp \left( - 10 \left(7.160(x-8.45)^2 + 5.925(x-8.45)(y-2.6) + 8.888(y-2.6)^2  \right)^{10} \right)
\\ &+ 
 \exp \left( - 10 \left(5.655(x-8.9)^2 -8.275(x-8.9)(y-2.6) + 14.34(y-2.6)^2  \right)^{10} \right)
 \Big) 
\\ + 
\left( 
\frac{14.6 - x}{18} 
\right)^{10} 
& \Big( 
 \exp \left( - 10 \left(1.140(x-9.2)^2 -0.229(x-9.2)(y-8.4) + 1.094(y-8.4)^2  \right)^{10} \right)
\\ &+ 
 \exp \left( - 10 \left(0.497(x-10.7)^2 -0.852(x-10.7)(y-8.6) + 1.392(y-8.6)^2  \right)^{10} \right)
\\ &+ 
 \exp \left( - 10 \left(3.305(x-12.0)^2 + 0.0(x-12.0)(y-8.25) + 0.64(y-8.25)^2  \right)^{10} \right)
\\ &+ 
 \exp \left( - 10 \left(8.163(x-12.1)^2 -1.482(x-12.1)(y-7.0) + 0.826(y-7.0)^2  \right)^{10} \right)
\\ &+ 
 \exp \left( - 10 \left(11.11(x-12.0)^2 -8.230(x-12.0)(y-5.8) + 4.938(y-5.8)^2  \right)^{10} \right)
\\ &+ 
 \exp \left( - 10 \left(4.711(x-12.4)^2 -4.266(x-12.4)(y-6.2) + 10.4(y-6.2)^2  \right)^{10} \right)
\\ &+ 
 \exp \left( - 10 \left(10.0(x-12.9)^2 -12.0(x-12.9)(y-6.4) + 10.0(y-6.4)^2  \right)^{10} \right)
 \Big) 
\\ + 
\left( 
\frac{3.1 - x + y}{10} 
\right)^{10} 
& \Big( 
 \exp \left( - 10 \left(1.777(x-10.1)^2 + 0.159(x-10.1)(y-11.5) + 2.366(y-11.5)^2  \right)^{8} \right)
\\ &+ 
 \exp \left( - 10 \left(1.777(x-11.2)^2 + -0.0(x-11.2)(y-11.5) + 2.777(y-11.5)^2  \right)^{8} \right)
\\ &+ 
 \exp \left( - 10 \left(4.0(x-12.2)^2 -3.085(x-12.2)(y-11.8) + 3.305(y-11.8)^2  \right)^{8} \right)
\\ &+ 
 \exp \left( - 10 \left(11.21(x-12.7)^2 -7.719(x-12.7)(y-12.4) + 3.106(y-12.4)^2  \right)^{10} \right)
\\ &+ 
 \exp \left( - 10 \left(6.25(x-12.85)^2 + 3.925(x-12.85)(y-13.55) + 3.305(y-13.55)^2  \right)^{8} \right)
 \Big) 
\\ + 
\left( 
\frac{6.82 - 2.2x + y}{3.0} 
\right)^{10} 
& \Big( 
 \exp \left( - 10 \left(3.305(x-8.4)^2 + 0.204(x-8.4)(y-12.75) + 1.777(y-12.75)^2  \right)^{8} \right)
\\ &+ 
 \exp \left( - 10 \left(4.340(x-8.75)^2 -3.700(x-8.75)(y-13.75) + 2.366(y-13.75)^2  \right)^{8} \right)
\\ &+ 
 \exp \left( - 10 \left(8.163(x-9.2)^2 -7.415(x-9.2)(y-15.1) + 2.601(y-15.1)^2  \right)^{8} \right)
\\ &+ 
 \exp \left( - 10 \left(25.0(x-9.6)^2 + 17.85(x-9.6)(y-16.5) + 6.25(y-16.5)^2  \right)^{8} \right)
 \Big) 
\\ &+ 
\frac{7.461 - x + y}{19.40} \exp \left( - 10 \left(0.029(x-15.0)^2 + 0.029(y-18.7)^2  \right)^{40} \right)
\\ &+ 
\frac{1}{2} \exp \left( - 10 e^{ 16\left( y - (100x^2 + 280x-3579)/2625 \right) } \right) 
\Bigg)^{1/10}
\end{align*}


\end{document}

















